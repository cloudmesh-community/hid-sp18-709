% status: 100
% chapter: Security

\title{Big Data Security and Privacy}

\author{Andres Castro Benavides}
\orcid{1234-5678-9012}
\affiliation{%
  \institution{Indiana University}
  \streetaddress{107 S. Indiana Avenue}
  \city{Bloomington}
  \state{Indiana}
  \postcode{43017-6221}
}
\email{acastrob@iu.edu}

\author{Uma M Kugan}
\affiliation{%
  \institution{Indiana University}
  \streetaddress{107 S. Indiana Avenue}
  \city{Bloomington}
  \state{Indiana}
  \postcode{43017-6221}
}
\email{umakugan@iu.edu}
% The default list of authors is too long for headers}
\renewcommand{\shortauthors}{Uma Kugan, Andres Castro}

\begin{abstract}
In recent history, the explosion of smart devices, social media
platforms and the Internet of Things has resulted in major 
replication of data.  This data is transferred at high speeds, 
in large volumes and in a variety of different platforms. Data is
one of the biggest assets of companies and with the exponential 
growth of data comes an increasing problem: Security and Privacy. 
Security and privacy have become a major initiative for every 
organization that has or utilizes databases.  These organizations
need to protect their brands, avoid penalties, and, in worst case
scenarios, avoid circumstances in which they may either lose a 
significant amount of business or their business as a whole. 
The computer science industry, alongside of these organizations, 
need to continue to develop ways to protect, utilize, and gain 
real-time insight from the data that each organization has. 
This paper is going to highlight security and privacy in Big Data, 
alongside of specific issues and challenges related to security.
\end{abstract}

\keywords{Security and Privacy, Big Data Security, Big Data, i523, 
HID305, HID323}

\maketitle

\section{Introduction}
Each organization has unique needs when it comes to Big Data.  
These needs cannot be described with one defined structure alone,
and likewise, the information that they use does not come with 
defined data types.  Because of this, there is the need for the
Big Data Platform. Big Data is gaining more popularity because 
of its ability to connect to a number of devices in the so-called
{\em Internet of Things} (IoT),  producing a huge dump of data that 
needs to be transformed into information assets. It is also very 
popular to buy additional on-demand computing power and storage
from public and private clouds to perform intensive data-parallel
processing. These things not only create the way for Big Data
expansion but also boosts security and privacy issues. Big Data
security is the process of securing data and their processes 
both within and outside the organization. Big Data deployments
are valuable targets for intruders and, because of this, security
becomes a never ending concern for any organization. A single
unauthorized user gaining access to an  organization's big data
could in and of itself acquire all the valuable information 
that the company possesses which could result not only in monetary
loss but also be detrimental to its business and to its brand name.
In current trends, security teams work towards continuously 
monitoring networks, hosts and application behavior across their
organization's data. Traditional methods of securing firewalls 
are no longer enough to secure a company's data assets and Big 
Data platforms need to be secured with a mix of both traditional
and newly developed security tools, as well as big data analytics
for monitoring security throughout the life of the 
platform~\cite{moura2016}.

\section{What is Big Data}
Big Data, by definition of its name, is an extensive variety and
heavy volume of data that can be entered or transferred at high
velocity, and include data sets coming from dynamic sources of
data and applies technologies to analyze these data sets. It is
a term usually used to define huge and complex data sets that do
not fit into any traditional system. Most recently, the term
{\em Big Data} tends to refer to the use of predictive, user behavior
analytics, or certain other advanced data analytics that extract 
value from data sets. These analytics provide more insights about
the data which indeed help businesses understand their trends
which will eventually, in good theory, help their growth
~\cite{hey2009fourth}.

For example, a company that works with waste management,
can collect data on  the waste production and human activities
from very diverse sources, then interpret the findings of
Big Data  to make optimal decisions~\cite{yenkar2014review}.

\section{Big Data Needs Big Security}
The amount of data collected by organizations and individuals 
around the world is growing on a daily basis, and the volume of
the data being collected is expected to continue to grow 
exponentially.  It is believed that the 90\% of the data we
have currently have in the world has been collected in the past
few years. Velocity, volume and variety of Big Data comes results
in privacy, security and compliance issues as well. Some of the
data stored in Big Data platforms is very sensitive and  regulations
need be put in place, strictly controlling specific aspects of the
data and  who has access to the data. Proper measures have to be
taken to control any weaknesses to cyber threats. 

There are requirements for security measures already in place. 
Big Data platforms are subject to compliance mandates by government
and industry regulations, including GDPR, PCI, Sarbanes-Oxley (SOX),
and HIPAA~\cite{imperva}. These measures place regulations on
company practices and implementations that ensure proper data
security and monitoring. These regulations are mandatory, and failing
to comply could result in severe penalties, from heavy fines to
legal actions.

While these requirements are important, traditional security
mechanisms that have been in place for securing structured 
static data are no longer sufficient. With technological 
advances also comes a need to continually assess weaknesses
in the new systems, to protect itself from new cyber threats
and hacking strategies, and to create user friendly platforms 
for client that do not compromise the data being collected
or stored. These developments are often far ahead of regulation, 
and individual entities need to be continually monitoring and
enhancing their platforms to ensure protection of its data 
and systems.  Big Data needs bigger security to protect its 
data, applications and infrastructure. Securing data not only 
protects the brand, reduces costs and avoids any legal issues, 
it also helps in retaining the brand name and increases 
revenue and growth~\cite{csasecurity}.

\section{Big Data Security Challenges}
Recent adoption of cloud storage has increased the amount of data
collected by organizations and hence it has become of vital 
importance to secure these data platforms as well. Data security 
issues are generally caused by the lack of proper tools and
measures provided by traditional anti-virus software. Routine
security checks to detect patches are no longer enough to 
handle real time influxes of data. Streaming real time data 
demands a great amount of attention focused on security and 
privacy solutions. Databases are no longer static. Big Data 
security's motto is to restrict unauthorized users and 
intruders from getting into a platform and also to block 
the encryption of data both in-transit and at-rest.  The
adoption of cloud storage creates a need to pay particular 
attention to the in-transit, or the continually expanding and 
modifying databases. Big Data security tools must be in place 
at all stages of data i.e. on incoming data, data stored in the 
platform and also on the data that goes out to other 
applications or outside party~\cite{datamation}.

\subsection{Access Control}
Access control, in the context of Big Data, is controlling who can
access data by using security settings. The different platforms 
that use Big Data need to be able to identify critical data,
data origination and also who has access to the data. In this
capacity, data access is not only protecting from external
access, but also protecting data from those who have 
internal access as well~\cite{rahmaniamathematical}.

User access should be controlled via a policy-based approach
that automates access based on user and role-based settings.
This manages different level of approvals in order to regulate
who has access to the critical data and to protect the
big data platform against inside attacks~\cite{dataconomy}.

\subsection{Audit Control}
Big Data analytics can be used to analyze different types of
logs in order to identify malicious activity. It also can
regularly audit all the working directories inside the 
organization in order to check for any unauthorized access
to any sensitive or privacy data. In reality, not all 
attacks are identified in the exact moment when the attack
occurs. In order to perform a root cause analysis of the 
incident, data security professionals need to have access 
to audit logs which allow them to trace attacks back to the
point of entry, exact time, modifications or weaknesses. 
In case of data breach, some firms are required to turn over
their audit logs to stakeholders and possibly affected 
companies and heavy fines are imposed for failure 
to comply~\cite{csasecurity}. 

\subsection{Real Time Compliance Control}
Real time security monitoring is always very challenging
due to the number of false positive alerts generated by
security programs.  Because of the frequency of false
positive alerts, they are usually ignored. Big Data analytics
may help provide more meaningful insights that could result 
in real time detection . 

\subsection{Non Relational Databases Privacy}
Non Relational Databases are still not fully matured. 
This poses a severe threat to securing the data and it is often
difficult for security and governance team to keep up with the 
demand. NoSQL databases primarily focus on how to handle high
volume of data without paying much attention to their
security needs. 

\subsection{End-Point Input Validation}
Many organizations collect their data from End-Point devices. 
It is very important to ensure that data coming from these
devices is not infected. Proper steps must be taken to make
sure data is coming from an authentic source and it is legitimate. 
Incoming data from End-Point devices such as smart phones is
growing tremendously and filtering or validating data from 
these sources is a very big challenge~\cite{csasecurity}. 

\subsection{Securing Transaction Logs and Data}
Data in any organization many be stored at various levels (tiers)
of the storage structure depending on the need and usage of
the data. Increase in the transfer of data within the organization
enforce for the need of auto-tiering for Big Data storage whereas
auto-tiering does not maintain the log of where the data is stored
and hence security is a big concern.

\subsection{Securing Distributed Framework}
Distributed framework enforces parallelism. This means that
data is distributed across multiple nodes to achieve faster
processing of large volumes of data. This increases the
security concern of the framework and the data that exists 
there. Most companies use a distributed framework like
MapReduce in which mappers read and compute and reducers
combine the output from each mapper. If mappers are not 
secured, there is the chance of data being 
compromised~\cite{dataconomy}.

\subsection{Data Provenance}
It is very important to know the original data that is coming
to the platform so that we can better classify them. Data
Origin should be consistently monitored but in reality 
due to the high volume it is becoming a big concern for 
data security. Provenance metadata is growing significantly 
as well and protecting metadata is very crucial 
for any organization~\cite{dataconomy}.

\section{Big Data Security Stakeholders}
In the digital era, the traditional way of securing the data,
changing passwords frequently, firewall protection is just not
enough to keep up with the growth of data produced by
Internet of Things(IoT), Smart Devices, Bring Your Own
Devices (BYOD) and several customer friendly apps that
is coming out everyday. ``Even though end user has the
biggest responsibility with securing his own data, unfortunately, 
end users are not fully aware of the cyber security issues 
and they do not have the appropriate knowledge to discover 
the world wide web in complete safety''~\cite{realdolmen}.

Big Data deployment is not possible to handle by any single 
business unit or with single tech team. It involves several
business units, infrastructure, information technology, 
security, compliance, programmers, testers and product
owners are all involved in big data deployment. They are 
all responsible for Big Data Security. Information Technology
and Security team is responsible for drawing the policies and 
procedures. Compliance officers together with security team
will protect compliance, such as automatically encrypting
personally identifiable information before it is easily 
accessible. Administrators will automate these process to
protect their environment. Even though every organizations
have their policies and control laid in place to protect their
biggest asset, phishing attacks can come in any form as a simple
email. Frequent internal audit within the company can help us
periodically check if all privacy, security and compliance
are all in place.If not, proper measures can be taken right
away to avoid any legal issues.

``The average annualized cost of cyber crime based upon a
representative sample of 237 organizations in six countries
by Ponemon Institute in their 2016 Cost of Cyber Crime Study
and the Risk of Business Innovation sponsored by Hewlett 
Packard Enterprise is 9.5million U.S. dollars''~\cite{ponemon}. 
In any organization, loss of information is the most expensive
consequence of a cyber crime. The cyber attack may results in
business disruptions, data or information loss, loss of revenue,
damage to equipment and last but not the least it damages 
the brand. So it is big time to protect and secure the big 
data and the environment from all angles.

\section{Best Practices for securing Big Data}
There are three fundamental principles used in defining 
security goals: confidentiality, integrity, and availability.
Confidentiality is the ability to keep sensitive data safe
from third parties and unauthorized access. Integrity in
this context means to avoid unauthorized modification of
the data. Finally, availability means always being able
to access the data and resources. These three concepts
are known as the CIA triad, and is used as base principles
when discussing and designing 
security practices~\cite{hamlin2016cryptography}.

To meet these goals, there are four main branches of 
security that apply to Big Data: Authentication, 
Encryption, Data Masking and Access Control~\cite{abouelmehdi201773}.

\subsection{Authentication}

Because of its nature (large sizes of data, linking different
sources, sharing access with third parties, etc), some of
Big Data's features are highly susceptible to different 
privacy, security and welfare risks~\cite{kshetri2014big}.

Privacy can be defined as the condition of confidentiality, 
protecting information from third parties. To support privacy,
there have been different Authentication methods that both
verify and validate entities who attempt to access the 
information. This ensures that only authorized entities 
are able to access the data or resources. 

With Big Data, it is important to choose a proper authentication
method, with the least computation complexity as possible,
to allow dynamic security solutions within large Data Centers
and also to avoid incrementing the traffic unnecessarily. 
Choosing an overbearing authentication method can cause both 
delay and storage issues.  Because of this, it is important 
to tailor the security to the needs of the
specific network~\cite{Thayananthan2015big}.

\subsection{Cryptography}

There are multiple understandings of how data moves through stages,
also known as Data Life cycles. Cryptography- define in terms
of security. CITA.  

From the perspective of cryptography, there are three phases
in the Data Life Cycle: Data in Transit, Data in Storage, 
and Data in Use.  Different cryptography techniques will be
implemented depending on which stage of the life cycle 
the data is in ~\cite{hamlin2016cryptography}.

There are different cryptographic tools that not only keep 
data secure at each point in its life-cycle, but also enable
richer use of the data. The main tool is Encryption.
Encryption takes pieces of data in plain text and use 
a cryptographic key to produce a version of the data that
can only be read using the cryptographic key. Without the
key, the information is illegible. There are two types of 
encryption: secret key encryption and public key encryption. 
Secret key encryption is when the same key is used for both 
encrypting and decrypting data. There are scenarios when one
of the keys can be made public. For instance, if the locking
key is kept private but the unlocking one is made public, 
this security can be used to prove
authenticity~\cite{hamlin2016cryptography}.

There are different standards for encryption. The most well known
and commonly used is Advanced Encryption Standard (AES). 
This standard sets guiding principles to ensure that data is
encrypted in a manner that meets security needs and allows
the recovery of original data~\cite{hamlin2016cryptography}.

\subsection{Data Masking}
By definition, Big Data works with large volumes of heterogeneous
data sets using software to manage the data and to provide
predictive analysis. Data masking works by replacing sensitive 
data with non-sensitive values, yet preserves the data integrity. 
For instance, replacing names with code names, or social security 
numbers with a key number. By doing this, different parties
can access information without putting sensitive data 
at risk~\cite{archana2017big}.

Five laws for data masking have been developed by Securosis 
Research. 
The first law is that data masking should not be reversible.
This means that the data should not be unmasked easily using
reverse engineering. 
The second law is that data that has been masked has to 
represent the original data set. For example, it has to belong
in the same context.
The third law states that data masking should maintain application 
and database integrity. This means that the process of data masking
should not modify or affect the data in the databases in a negative way. 
The fourth law emphasizes that non-sensitive data can be masked,
but it should not be masked if it can not be used to make sensitive
data vulnerable. For instance, when masking information about a 
person, it is correct to mask the person’s name, email address 
and social security number, but other information like gender,
or favorite colour, would be useless to mask.
Finally, data masking must be a repeatable process, using a standard 
to reproduce the steps taken to mask the data, allowing to troubleshoot 
possible problems in the process~\cite{archana2017big}.


\subsection{Access Control}
As it was explained in the Challenges section, Access control,
allows some entities to access the data or resources, while denying
its use to other entities. Through security settings. 

Some authors add that the inferences drawn from data should also be a 
cause for concern, because they can identify traits and patterns that
could  expose vulnerabilities. They propose that organizations who
use the protected data should disclose their decisions criteria in
order to apply access control in a broader spectrum. By doing so, 
it would be sufficient to diminish privacy concerns by de-identifying
the data, or denying access to certain parts of the data that coul
be used to make entities or data vulnerable. Some of these authors 
say that by doing this, it would not only reduce the privacy risk, 
it would also salvage large amounts of data for alternative use.
This de-identification can also be achieved through data masking,
pseudonymization, aggregation, among other methods~\cite{tene2012big}.

\subsection{Physical Security}
It is always better to build and deploy Big Data platforms in their 
own data center. If deployed in a cloud, the organization must 
diligent to ensure that the cloud provider's data center is physically
well secured. Access should be restricted to strangers and staff 
who have no official responsibilities in the designated areas or
interacting with the data sources. Data centers should be properly
monitored at all times and video surveillance and security logs 
are important tools to achieve this.  

\section{Future of Big Data Security}
To think about Future of Big Data Security, it is necessary to 
engage the conversation of what the trends are in Big Data
and what technologies are expanding and changing the horizon.
There are many new technologies and solutions that are shaping 
the future of the Big Data, but because of the length and focus
of this document, there are three main areas that will be
covered: Virtualization and Cloud Computing, IOT Security,
External Password Vaults and Penetration Tests.

\subsection{Virtualization and Cloud Computing}

Virtualization is a way of deploying resources at multiple levels, 
such as hardware, network infrastructure, application and desktop 
centralized managing and using dynamically the physical resources.
This makes the system flexible and less costly than traditional 
environments and giving management new tools to optimize the
use of resources~\cite{padmini2015securing}.

Since virtualization can be developed in so many levels, including
cloud computing and by multiple service providers, it is 
natural that the system requirements of users and organizations
move towards a variety of solutions that may include 
Infrastructure as a Service (IaaS) frameworks from public 
clouds such as the ones offered by Amazon, Microsoft, Google,
Rackspace, HP, among others, or even Private clouds,
maintained and many times even set up by internal IT
departments~\cite{von2014accessing}.

These cloud computing technologies are being used to solve
data-intensive problems on large-scale infrastructure. Thus,
integrating big data technologies and cloud computing for data
mining, knowledge discovery, and decision-making~\cite{kune2016anatomy}.

\subsection{IOT Security}

The Internet of Things (IoT) is the name given to the large network
of physical devices that does not match the typical concept of 
computer networks, this includes all kinds of objects. The large
and growing amount of devices and diverse uses given to them, 
makes IoT generates very important Big Data streams. Making it 
necessary to develop new systems and data mining techniques
for this new paradigm~\cite{bifet2016mining}.

In this IoT paradigm, each new opportunity opens doors to
new threats as well. This makes it necessary to develop
techniques to ensure trust, security and privacy. 
Different Authors write about the possible ways to
face these challenges, and some, they consider three
main axes to articulate the solutions: Effective 
security - used in very small embedded networks,
context-aware privacy and user-centric privacy, and the
third one is the systemic and cognitive approach for
IoT security - where the interaction between people
and the IoT can be envisioned as a set of nodes and 
tensions~\cite{riahi2014systemic}.

All this to say that in order to approach privacy and 
security in this new paradigm, many new theories and techniques
have been developed since old security products and techniques 
may not suffice the needs of the different IoT users and
communications.


\subsection{External Password Vaults}

Password vaults are applications that store multiple passwords 
and encrypt them storing them in a 
database~\cite{chatterjee2015cracking}.

There are small Password Vaults that can be stored locally
on a system, or larger options that can be integrated into
larger systems, providing additional security options, like
generating real time temporary passwords for effective
password rotation (I.E. Cyberark External Password Vault)
~\cite{nelson2015practical}.

These techniques are key to articulate authentication and a 
proper data access while using multiple services such as 
Cloud infrastructure and IoT.

\subsection{Penetration Tests}

After applying all the security techniques and strategies, and
after putting in place all necessary security and privacy 
policies, the most important step is validating the strength
of the security of the system. For some time, companies have
started to perform tests that consist on simulating an attack
from the perspective of an attacker, this method is known as
Penetration test and it allows to actively evaluate and assess 
the security of a system~\cite{shivayogimathoverview}.

The  tester identifies the threats faced by an organization from
hackers and suggest changes to improve the security and minimize
the vulnerabilities and close the possible loop holes in the
network~\cite{shivayogimathoverview}.


\section{Conclusions}

Big Data as a constantly evolving and ever changing branch of
information technologies resembles an ecosystem that since
it covers gathering data from so many sources, processing
it and generating new information, there will be many entities 
and interests involved that will need to be protected. The 
features of Big Data such as Volume, Variety and Velocity
bring new challenges to security and privacy protection. 
To protect the integrity and availability, security providers
and local IT departments, will have to diversify their security
and privacy strategies and policies, in order to keep pace with
the growth and evolution of this new ecosystem.

\begin{acks}
The authors would like to thank Dr.~Gregor~von~Laszewski for his 
support and suggestions in writing this paper.
\end{acks}

\bibliographystyle{ACM-Reference-Format}
\bibliography{report}

\newpage

\appendix

\section{Work Breakdown}

\begin{description}

\item [Uma Kugan] Research for Section Big Data Needs Big Security, Big Data Security and Challenges.

\item[Andres Castro Benavides] Research for Section Best practices and Future

\item[Editing:] Andres Castro Benavides and Uma Kugan

\end{description}


